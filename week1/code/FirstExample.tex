\documentclass[12pt]{article}

\title{A very simple document on the Metabolic Theory of Ecology}

\author{Vitor Ferreira}

\date{}

\begin{document}
  \maketitle
  
  \begin{abstract}

    This paper elaborates on the principles of chemistry, physics and biology drawn into a fundamental equation that links indididuals to higher orders of organization.
  \end{abstract}
  
  \section{Introduction}

    Metabolism is the rate that organisms uptake and allocate resources for their growth, survival and reproduction \cite{brown2004toward}.
  \section{Materials \& Methods}

    Accounting for the effects of body size and temperature on individual metabolic rates, the formula can be provided as follows:
  
  \begin{equation}
    I = i_0 * M^\frac{3}{4} e^\frac{-E}{kT}
  \end{equation}

    where, $i_0$ is the normalization constant, $M$ is the body size, $E$ is the activation energy, $k$ the Boltzmann's constant and $T$ is the absolute temperature in K.
    
    The allometric scaling of this phenomenum applies across several orders of magnitude and levels of organization.
  
  \bibliographystyle{plain}
  
  \bibliography{FirstBiblio}

\end{document}